\documentclass[11pt]{article}
\usepackage[top=2cm,bottom=2cm,left=2cm,right=2cm]{geometry}
\geometry{a4paper}
\usepackage[T2A]{fontenc}
\usepackage[utf8]{inputenc}
\usepackage[russian]{babel}
\usepackage{amssymb}
\usepackage{amsmath}
\usepackage{graphics}
\usepackage{subcaption}
\usepackage{multirow}

\begin{document}
\section*{Введение}
Одним из основопологающих принципов классической механики является принцип локального реализма. Под локальностью здесь понимается условие, что на объект может оказывать влияние только его ближайшее окружение, так как никакое действие не может передаваться от одной частицы к другой быстрее скорости света. Под реализмом в физике подразумевается философское предположение, что все объекты обладают "объективно существующими" значениями всех своих параметров, которые только могут быть измерены, еще перед тем как проводятся сами измерения. Объединение этих двух постулатов не только не противоречат классической механике и общей теории относительности, но и интуитивно кажется правдивым предположением об устройстве реальности.

Однако данные современной квантовой механики ставят под сомнение адекватность модели локального реализма для описания микромира. В соответствии с принципом неопределенности Гейзенберга существует фундаментальный предел точности для совместного измерения значений кооординаты квантовой частицы и ее импульса(как и для любого другого совместного измерения двух наблюдаемых, описываемых некоммутирующими операторами). Это, в свою очередь, означает, что значения координаты и импульса не могут считаться определенными до проведения измерения, так как измерение одной величины вносит неустранимое возмущение в состояние частицы, что приводит к искажению при измерении второго параметра. Таким образом, следствием принципа неопределенности является, что в квантовом мире нет места реализму.

Однако, в статье \cite{EPR} Ейнштейн, Подольский и Розен показали, что, используя формализм квантовой мехнакики, можно придти к противоречию с вышеупомянутым принципом. Они представили мысленный эксперимент, в ходе которого можно точно определить и координату, и импульс одной частицы, проводя измерения для частиц в сцепленном состоянии, находящихся на удалении друг от друга. Решение этого парадокса подняло вопрос о возможной неполноте квантовой теории. Возможно, квантовая механика не полностью описывает состояние системы, и существуют еще некие неизвестные скрытые параметры. Другим объяснением парадокса является отказ от принципа локальности.

Еще одно свидетельство о противоречиях между классической и квантовыми моделями было получено Бэллом(а в последствии и другими авторами) в виде статистических неравенств, которые могут быть проверены экспериментально. Если в начале вопрос о существовании локальности и реализма был скорее философским рассуждением о природе реального мира, то с помощью этих неравенств он был сформулирован математически и протестирован в экспериментах. Первая экспериментальная проверка неравенства была осуществлена Фридманом и Клозером в 1972 году, и после этого подобные опыты проводились неоднократно. Несмортя на то, что нарушения неравенств Бэлла, предсказываемые квантовой механикой, наблюдались в большинстве экспериментов, процесс их проверки все еще не завершен.

Одной из причин продолжения экспериментов по этой теме является то, что в любом подобном опыте эффективность детекторов составляет меньше 100\%, кроме того, есть некоторое количество ложных срабатываний. Во многие неравенствах типа Бэлла подразумевается отсутствие этих факторов. Примером неравенства, учитывающего эти типы ошибок, является неравенство Эберхарда, представленное в статье \cite{Eberhard}. На его основе был проведен ряд экспериментов \cite{Zeilinger}, также выявивших нарушение предсказаний классической теории.

Основной трудностью в проведении таких экспериментов является то, что для нарушения неравенства требуется достаточно высокая эффективность детекторов и специальное начальное состояние квантовой системы, а также параметры установки детекторов, задаваемые углами. Эти параметры должны быть такими, чтобы неравенство нарушалось как можно сильнее. Для достижения этого результата Эберхард использовал простую оптимизацию перебором.

Данная работа посвящена математическому моделированию параметров неравенства Эберхарда с использованием методов оптимизации. Основной целью работы является рассмотрение более общего случая, когда детекторы имеют различные эффективности. Кроме того, в статье \cite{Eberhard} в качестве целевой функции рассматривается величина, имеющая смысл математического ожидания. Однако, как и для любой статистической величины, интерес представляет также возможная степень разброса результатов, выражаемая через величину стандартного отклонения. В данной работе рассматривается также оптимизация параметров неравенства Эберхарда для коэффициента вариации с учетом возможных погрешностей при установке углов детекторов в ходе эксперимента.

\section*{Заключение}
В ходе выполения работы была рассмотрена модель Эберхарда, описывающая предсказания квантовой механики для случая одинаковых эффективностей детекторов. Для этой модели была проведена оптимизация параметров, результаты которой согласуются с аналогичными результатами из рассматриваемой статьи. Аналогичная процедура была проведена для случая различных эффективностей.

Кроме того, было высказано предположение, что для оценки степени нарушения необходимо использовать целевую функцию, учитывающую величину возможного разброса экспериментальных данных, выраженную через коэффициент вариации $\sigma_J / J$. Было показано, что в модели, описанной Эберхардом, минимальное значение математического ожидания достигается для $\psi$, равного собственному вектору матрицы $\mathcal{B}$, и дисперсия в этом случае равна нулю.

Однако, в случае, если в модель добавляется погрешность установки детекторов, выраженная через погрешности углов, то эта целевая функция позволяет найти такие параметры, при которых неравенство будет нарушаться даже с учетом возможных погрешностей. Полученнные параметры отличаются от результатов Эберхарда.

Кроме того, была рассмотрена реализация тестирования неравенства Эберхарда из статьи \cite{Zeilinger}. Для параметров, указанных там, при заданных значениях эффективностей была также проведена оптимизация. При подстановке полученных значений в модели Цайлингера наблюдается более сильное нарушение неравенства, чем указанное в статье.

Полученные результаты позволяют ожидать, что при проведении экспериментов с найденными значениями будут получены результаты, нарушающие неравенство Эберхарда даже с учетом различных эффективностей детекторов и погрешностями в установке углов.

\bibliography{literature.bib}
\bibliographystyle{ieeetr}
\end{document}