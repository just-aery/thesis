\documentclass[11pt]{article}
\usepackage[top=2cm,bottom=2cm,left=2cm,right=2cm]{geometry}
%\geometry{a4paper}
\usepackage[T2A]{fontenc}
\usepackage[utf8]{inputenc}
\usepackage[russian]{babel}
\usepackage{amssymb}
\usepackage{amsmath}

\begin{document}
\paragraph{Слайд 1}
Добрый день, уважаемая комиссия. Вашему вниманию предлагается работа на тему: «Оптимизация статистических параметров для неравенства Эберхарда»
\paragraph{Слайд 2}
Проблема, является ли квантовая механика наболее полным из возможных описаний реальности микромира или существует более полная классическая вероятностная теория(теория скрытых параметров), остается открытым в течение уже длительного времени.

Рассматриваемая в данной работе задача имеет отношение к экспериментальной проверке статистических неравенств, имеющая целью установить возможность существования классической модели, согласующейся с предсказаниями квантовой механики.

Одно из таких неравенств - неравенство Эберхарда. В его модели учитываются показатели эффективности детектирования и возможные ложные срабатывания, что делает его удобным для экспериментальной проверки. Интерес представляет максимальное нарушение неравенства, поэтому для подбора параметров имеет смысл использовать оптимизацию.

\paragraph{Слайд 3}
Цели работы:
\begin{enumerate}
\item  Оптимизация параметров таким образом, чтобы наблюдалось наиболее сильное нарушение неравенства.
\item Так как в реальных экспериментах почти всегда используются 2 детектора с существенно отличающимися эффективностями, необходимо получить параметры их установки для этого случая.
\item Учесть величину разброса данных - стандартного отклонения в целевой функции. Исследовать влияние погрешности при установке значений параметров.
\end{enumerate}

\paragraph{Слайд 4}
Рассмотрим некоторые основные положения квантовой механики, используемые при формулировке модели. Квантовое состояние - вектор комплексного гильбертового пространства, а все физические наблюдаемые - самосопряженные операторы из этого пространства. Спектр значений наблюдаемой совпадает со спектром оператора. Любое значение из спектра может быть получено с некоторой вероятностью в ходе измерения.

После измерения квантовое состояние системы проецируется на собственное подпространство, соотвенствующее измеренному собственному значению. Для объединенной системы пространство состояний - тензорное произведение пространств отдельных систем.

\paragraph{Слайд 5}
Ансамбль состояний описывается через оператор плотности, где каждое состояние реализуется с некоторой вероятностью. 

В квантовой механике значения мат. ожидания и дисперсии для наблюдаемой зависят также от состояния системы до измерения, которое в общем случае выражено через оператор плотности $\rho$. Формулы для их нахождения представлены на слайде.

\paragraph{Слайд 6}
Рассматриваемый эксперимент состоит в следующем. Берется источник, испускающий фотоны в сцепленном состоянии. На довольно большом удалении друг от друга устанавливаются поляризационные призмы, после которых стоят детекторы. Для первой системы угол установки призмы обозначим через $\alpha$, для второй - через $\beta$. всего проводится четыре сеанса измерения - по 2 варианта установки угла призмы. Для каждого отдельно взятого фотона возможны 3 варианта детектирования - в обыкновенном, необыкновенном луче на выходе из призмы либо отсутствуе срабатывания детектора.

Само неравенство Эберхарда представлено на слайде. $n$ здесь - количество раз, когда при указанных углах для частиц наблюдалась указанная в нижнем индексе комбинация результатов детектирования.

\paragraph{Слайд 7}
Количество незадетектированных фотонов определяется эффективностью детектора. Так как возможны ложные срабатывания из-за фонового шума, в модели также присутствует коэффициент фонового шума $\zeta$. Вид неравенства с учетом этого представлен на слайде.

Значения величин $n$ брались из предсказаний квантовой механики для мат. ожидания результатов детектирования поляризации. На этом и последующем слайде представлены формулы, по которым проводились расчеты. Все расчеты проводились в относительных величинах, то есть не учитывалось число экспериментов $N$.

\paragraph{Слайд 8}
Так как значение уровня шума вносит константный вклад в целевую функцию, его можно не учитывать при оптимизации. Примем значения разностей углов установки призм равными и будем использовать указанное квантовое состояние.

Таким образом, для нахождения наибольшего нарушения неравенства при заданных эффективностях необходимо варьировать 3 параметра - $r, \omega, \theta$.

\paragraph{Слайд 9}
На этом слайде представлена часть результатов оптимизации параметров для случая различных эффективностей. Первый график - величина $\theta$, второя - значения целевой функции. Случай различных эффективностей согласуется с результатами самого Эберхарда, согласно которым, нарушение неравенства наблюдается только начиная с $\eta = 0.67$. И чем больше эффективность, тем сильнее нарушение. С точки зрения разницы эффективностей детекторов, не существует оптимальной, чем выше каждая из величин в отдельности, тем лучше результат.

\paragraph{Слайд 10}
Этот подход оптимизации параметров был проверен на результатах из другой статьи авторства Цайлингера, где помимо предсказаний модели приводятся экспериментальные результаты. Основное отличие его модели - другое квантовое состояние. Также устанавливаются все 4 угла в отдельности. После проведения оптимизации были найдены параметры, которые дают большее нарушение неравенства, чем то, что указано в статье.

\paragraph{Слайд 11}
Для целевой функции $J$, выражающей мат. ожидание, справедливо следующее утверждение. Квантовое значение, которое ее минимизирует. является собственным вектором матрицы $\mathcal{B}$. В этом случае квантовая дисперсия равно нулю. Это утверждение следует из теоремы Куранта-Фишера. В случае нормированных состояний мат. ожидание принимает значения из промежутка от самого маленького до самого большого собственного значения. Минимуму соответствует состояние, являющееся собственным вектором. Легко показать, что для собственного вектора дисперсия равна нулю. 

Это означает, что даже с учетом возможного разброса значений экспериментов, выраженного через величину стандартного отклонения, это состояние является оптимальным.

\paragraph{Слайд 12}
Рассмотрим случай, когда значения углов не могут быть выставлены точно. Примем, что реальная величина разности углов равномерно распределена на отрезке вокруг желаемого значения. В этом случае значения мат. ожиданий и дисперсий изменятся.

Рассмотрим целевую функцию, которая будет учитывать относительный разброс получаемых значений. Она выражается через коэффициент вариации $J / \sigma$.

\paragraph{Слайд 13}
Результаты оптимизации для этого случая представлены на слайде. Так как реальные значения эффективностей - около $\eta = 0.85$, то из второго графика видно, что значения мат. ожидания не сильно отличаются от случая нулевой эффективности. Кроме того, для различных значений $\delta$ в пределах одного градуса графики почти совпадают.

\paragraph{Слайд 14}
Усложним модель разброса - будем рассматривать все углы в отдельности. Тогда значения дисперсии и мат. ожидания принимают вид, см. слайд.

\paragraph{Слайд 15}
Результаты оптимизации для этого случая представлены на слайде. Из графиков видно, что добавление разброса почти не изменяет оптимальные параметры. А это означает, что можно сделать предположение, что контроль за величиной углов детекторов можно уменьшить.

\paragraph{Слайд 16}
Таким образом, были получены следующие результаты:
\begin{enumerate}
\item Были найдены параметры для случая различающихся эффективностей.
\item Оптимизация позволила получить более сильное нарушение неравенства для параметров из статьи Цайлингера.
\item Была проведена оптимизация с учетом возможных погрешностей установки углов. Она выявила, что значения нарушения в этом случае мало меняются.
\end{enumerate}

\end{document}