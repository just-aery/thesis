\documentclass[11pt]{article}

\usepackage[T2A]{fontenc}
\usepackage[utf8]{inputenc}
\usepackage[english, russian]{babel}
\usepackage{amssymb}
\usepackage{amsmath}

\usepackage[colorlinks=true]{hyperref}
%\usepackage[top=2cm,bottom=2cm,left=2cm,right=2cm]{geometry}

\usepackage{amsthm}
\newtheorem{theorem}{Теорема}[section]
\newtheorem{definition}{Определение}[section]

%\newenvironment{proof}[1][Proof]{\begin{trivlist}
%\item[\hskip \labelsep {\bfseries #1}]}{\end{trivlist}}
%\newenvironment{definition}[1][Definition]{\begin{trivlist}
%\item[\hskip \labelsep {\bfseries #1}]}{\end{trivlist}}
%\newenvironment{theorem}[1][Theorem]{\begin{trivlist}
%\item[\hskip \labelsep {\bfseries #1}]}{\end{trivlist}}
%\newenvironment{example}[1][Example]{\begin{trivlist}
%\item[\hskip \labelsep {\bfseries #1}]}{\end{trivlist}}
%\newenvironment{remark}[1][Remark]{\begin{trivlist}
%\item[\hskip \labelsep {\bfseries #1}]}{\end{trivlist}}
%
%\newcommand{\qed}{\nobreak \ifvmode \relax \else
%      \ifdim\lastskip<1.5em \hskip-\lastskip
%      \hskip1.5em plus0em minus0.5em \fi \nobreak
%      \vrule height0.75em width0.5em depth0.25em\fi}

\usepackage{enumitem}

\DeclareMathOperator{\Tr}{Tr}


\begin{document}

\section{Introduction}
Одним из основопологающих принципов классической механики является принцип локального реализма. Под локальностью здесь понимается условие, что на объект может оказывать влияние только его ближайшее окружение, так как никакое действие не может передаваться от одной частицы к другой быстрее скорости света. Под реализмом в физике подразумевается философское предположение, что все объекты обладают "объективно существующими" значениями всех своих параметров, которые только могут быть измерены, еще перед тем как проводятся сами измерения. Объединение этих двух постулатов не только не противоречат классической механике и общей теории относительности, но и интуитивно кажется правдивым предположением об устройстве реальности.

Однако данные современной квантовой механики ставят под сомнение адекватность модели локального реализма для описания реальности. В соответствии с принципом неопределенности Гейзенберга существует фундаментальный предел точности для совместного измерения значений кооординаты квантовой частицы и ее импульса(как и для любого другого совместного измерения двух наблюдаемых, описываемых некоммутирующими операторами). Это, в свою очередь, означает, что значения координаты и импульса не могут считаться определенными до проведения измерения, так как измерение одной величины вносит неустранимое возмущение в состояние частицы, что приводит к искажению при измерении второго параметра. Таким образом, следствием принципа неопределенности является, что в квантовом мире нет места реализму.

Однако, в статье \cite{EPR} Ейнштейн, Подольский и Розен показали, что, используя формализм квантовой мехнакики, можно придти к противоречию с вышеупомянутым принципом. Они представили мысленный эксперимент, в ходе которого можно точно определить и координату, и импульс одной частицы, проводя измерения для частиц в сцепленном состоянии, находящихся на удалении друг от друга. Решение этого парадокса подняло вопрос о возможной неполноте квантовой теории. Возможно, квантовая механика не полностью описывает состояние системы, и существуют еще некие неизвестные скрытые параметры. Другим объяснением парадокса является отказ от принципа локальности.

Еще одно свидетельство о противоречиях между классической и квантовыми моделями было получено Бэллом(а в последствии и другими авторами) в виде статистических неравенств, которые могут быть проверены экспериментально. Если в начале вопрос о существовании локальности и реализма был скорее философским рассуждением о природе реального мира, то с помощью этих неравенств он был сформулирован математически и протестирован в экспериментах. Первая экспериментальная проверка неравенства была осуществлена Фридманом и Клозером в 1972 году, и после этого подобные опыты проводились неоднократно. Несмортя на то, что нарушения неравенств Бэлла, предсказываемые квантовой механикой, наблюдались в большинстве экспериментов, процесс их проверки все еще не завершен.

Одной из причин продолжения экспериментов по этой теме является то, что в любом подобном опыте эффективность детекторов составляет меньше 100\%, кроме того, есть некоторое количество ложных срабатываний. Примером неравенства, учитывающего эти типы ошибок, является неравенство Эберхарда, представленное в статье \cite{Eberhard}. На его основе был проведен ряд экспериментов \cite{Zeilinger}, также выявивших нарушение предсказаний классической теории.

Данная работа посвящена математическому моделированию параметров неравенства Эберхарда, которые используются экспериментаторами для настройки углов расположения детекторов.

\section{Необходимые элементы функционального анализа и теории обобщенных функций}
\subsection{Гильбертово пространство и линейный оператор}
Рассмотрим некоторые определения из функционального анализа, которые используются для формального описания квантовой механики. Более подробные сведения могут быть найдены, например в \cite{advanced_la}.


\begin{definition}
Скалярным произведением $\langle\cdot,\cdot\rangle$ в комплексном векторном пространстве $V$ называется отображение $V\times V \to \mathbb{C}$, удовлетворяющее следующим условиям:
\begin{enumerate}
\item Сопряженная симметрия - $\langle x, y\rangle = \overline{\langle y, x\rangle}$,
\item Положительная определенность - $\langle x, y\rangle \geq 0$ и $\langle x, y\rangle = 0 \Leftrightarrow x = y$,
\item Линейность по второму аргументу - $\langle x, ay\rangle = a\langle x, y\rangle$, где $a\in\mathbb{C}$, и $\langle x, y_1 + y_2\rangle = \langle x, y_1\rangle + \langle x, y_2\rangle$.
\end{enumerate}
\end{definition}

\begin{definition}
Гильбертово пространство $H$ - это комплексное векторное пространство с определенным в нем скалярным произведение, которое является полным относительно метрики, порожденной этим скалярным произведением $\|x\| = \sqrt{\langle x, x\rangle}$.
\end{definition}

Пространство $H = \mathbb{C}^n$ со скалярным произведением $\langle\psi, \varphi \rangle = \sum_{i = 1}^n \overline{\psi_i}\varphi_i$ является гильбертовым. Другой пример гильбертова пространства - пространство $H = L_2(\mathbb{R}^n, dx)$ квадратично интегрируемых по Лебегу функций. Скалярное произведение в этом случае определяется как $\langle\psi, \varphi \rangle = \int_{\mathbb{R}^n}\overline{\psi(x)}\varphi(x) dx$.

Другим часто используемым в квантовой механике понятием является понятие линейного оператора.
\begin{definition}
Линейный оператор $A$ в гильбертовом пространстве $H$ - это отображение $H \to H$, обладающее следующими свойствами :
\begin{enumerate}
\item $\forall \psi_1,\psi_2\in H: A(\psi_1 + \psi_2) = A\psi_1 + A\psi_2 $,
\item $\forall\lambda\in\mathbb{C}, \psi\in H: A(\lambda\psi) = \lambda A\psi$.
\end{enumerate}
\end{definition}

\begin{definition}
Линейный оператор $A^*$ называют сопряженным к линейному оператору $A$, если для любых векторов $\varphi, \psi$ верно равенство $\langle A\psi,\varphi\rangle = \langle\psi, A^*\varphi\rangle$. Оператор $A$ называется самосопряженным, если $A^* = A$.
\end{definition}

\begin{definition}
Рассмотрим векторное пространство $V$ над полем комплексных чисел $\mathbb{C}$ и линейный оператор $A$ в этом пространстве.
\begin{enumerate}
\item Величина $\lambda\in \mathbb{C}$ называется собственным значением $A$ i, если существует ненулевой вектор $\psi\in V$, для которого
\[
A\psi = \lambda\psi.
\]
\item Такой вектор $\psi$ называется собственным вектором оператора $A$.
\item В некоторых случаях несколько собственных векторов соответствуют одному собственному значению $\lambda$. Множество таких векторов вместе с нулевым вектором образуют подпространство пространства $V$, называемое собственным подпространством для данного $\lambda$.
\item Множество всех собственных значений оператора называется его спектром.
\end{enumerate}
\end{definition}

Для любого самосопряженного оператора справедлива следующая теорема. 
\begin{theorem}
Для любого самосопряденного оператора $A$ в комплексном гильбертовом пространстве $H$ выполняются утверждения:
\begin{enumerate}
\item Все собственные значения $A$ являются действительными,
\item В $H$ существует ортонормированный базис, образованный из собственных векторов оператора.
\end{enumerate}
\end{theorem}

Другой часто используемый оператор в гильбертовом пространстве - оператор проектирования.
\begin{definition}
Рассмотрим $H_0$ - линейное подпространство $H$. Оператор $\pi: H \to H - H_0$ называется проектором, если $\pi^* = \pi$ и $\pi^2 = \pi$. Он проектирует $H$ перпендикулярно на $H_0 = \pi(H)$.
\end{definition}
Обозначим через $\{e_1, \ldots, e_n\}$ базис в $H$ и через $\{e_1, \ldots e_m\}$ базис в $H_0$. Если $e_j \in H_0$, то $\pi e_j = e_j$, а если $e_j\in H_0^\perp$, тогда $\pi e_j = 0$. Для произвольного элемента $\psi\in H$ оператор проектирования действует следующим образом: 
\[
\pi\psi = \sum_{i = 1}^m z_ie_i, \mbox{ where } z_i = \langle e_i, \psi\rangle.
\]

\subsection{Тензорное произведение}
Рассмотрим векторное пространство $H_1$ с ортогональным базисом $\{e_1, \ldots e_n\}$ и векторное пространство $H_2$ с ортогональным базисом $\{f_1,\ldots f_m\}$. С помощью формального символа $\otimes$ обозначим элементы нового ортогонального базиса, сформированного из множества упорядоченных пар $(e_i, f_j)$ в новом векторном пространстве:
\begin{equation}
\{e_i \otimes f_j | e_i \in H_1, f_j \in H_2\}. \label{eq:tensor_basis}
\end{equation}

\begin{definition}
Множество формальных сумм вида
\begin{equation}
\psi = \sum_{i,j}z_{i,j}e_i\otimes f_j,\ \sum_{i, j}|z_{i,j}|^2 < \infty,\  z_{i,j} \in \mathbb{C}
\label{eq:tensor_element}
\end{equation}
с естественно определенными операциями сложения и умножения на скаляр называется тензорным произведением $H_1$ и $H_2$ и обозначается через $H = H_1\otimes H_2$.
\end{definition}

Если пространства $H_1$ и $H_2$ являются гильбертовыми, тогда $H = H_1\otimes H_2$ также является гильбертовым пространством со скалярным произведением, заданным с помощью выражения
\[
\langle\psi, \varphi\rangle = \langle \sum_{i,j}z_{i,j}e_i\otimes f_j, \sum_{k,l}v_{k,l}e_k\otimes f_l \rangle = \sum_{i, j}\sum_{k, l}\overline{z_{i,j}}v_{k,l} \langle e_i\otimes f_j, e_k\otimes f_l \rangle,
\]
где скалярное произведение базисных векторв определено как
\[
\langle e_i\otimes f_j, e_k\otimes f_l \rangle \overset{\mbox{def}}= \langle e_i, e_k\rangle \cdot \langle f_j, f_l \rangle = \delta_{i, k}\delta_{j, l}.
\]

Определим тензорное произведение элементов $\psi_1, \psi_2$ пространств $H_1$ и $H_2$, соответственно:
\begin{equation}
\psi_1 \otimes \psi_2 = \left(\sum_ix_ie_i\right) \otimes \left(\sum_jy_jf_j\right) \overset{\mbox{def}}= \sum_{i, j} x_iy_j \cdot e_i \otimes f_j \in H_1 \otimes H_2.
\label{eq:tensor_factorizable}
\end{equation}
Элементы из $H = H_1\otimes H_2$, которые могут быть представлены в форме \eqref{eq:tensor_factorizable} называются факторизуемыми. Однако, пространство $H$ не состоит только из факторизуемых элементов. Приведем пример вектора, не представимого в виде \eqref{eq:tensor_factorizable}.

Пусть $H = H_1\otimes H_2$ - тензорное произведение гильбертовых пространств $H_i$ с базисом $\{e_1, e_2\}$. Покажем, что следующий вектор из $H$ не является факторизуемым: 
\[
	\psi = \frac{e_1 \otimes e_2 + e_2 \otimes e_1}{2} \ne \psi_1 \otimes \psi_2
\]
для любых пар $\psi_1, \psi_2$.
\begin{proof}
Предположим, что $\psi$ представим в виде $\psi = \psi_1\otimes\psi_2$, где 
$\psi_1$ и $\psi_2$ могут быть записаны как взвешенная сумма базисных векторов:
\begin{gather*}
	\psi_1 = a_1 e_1 + a_2 e_2,\ a_i\in \mathbb{C},\ |a_1|^2 + |a_2|^2 = 1 \\
	\psi_2 = b_1 e_1 + b_2 e_2, b_i\in \mathbb{C}, |b_1|^2 + |b_2|^2 = 1.
\end{gather*}
Для такого $\psi$ имеем:
\begin{eqnarray*}
	\psi_1 \otimes \psi_2 &=& (a_1 e_1 + a_2 e_2) \otimes (b_1 e_1 + b_2 e_2) = \\ &=&  
	a_1 b_1 \cdot e_1 \otimes e_1 + a_1 b_2 \cdot e_1 \otimes e_2 + 
	a_2 b_1 \cdot e_2 \otimes e_1 + a_2 b_2 \cdot e_2 \otimes e_2.
\end{eqnarray*}
Чтобы найти коэффициенты $a_1$, $a_2$, $b_1$, $b_2$, необходимо решить следующую систему уравнений:
$$
\begin{cases}
a_1 b_2 = \frac12 \\
a_2 b_1 = \frac12 \\
a_1 b_1 = 0 \\
a_2 b_2 = 0
\end{cases},
$$
которая является несовместной. Поэтому предположение неверно и $\psi$ не являетс факторизуемым.
\end{proof}
 
\begin{definition}
Рассмотрим два линейных оператора $A_1: H_1\to H_1$ и $A_2: H_2\to H_2$. Их тензорное произведение задается выражением
\[
(A_1\otimes A_2)(\psi_1\otimes \psi_2) = A_1\psi_1 \otimes A_2\psi_2.
\]
%In case of operators $A_1, A_2$ are written in a matrix form as $A^1, A^2$ correspondingly their tensor product is defined by a Kronecker product of two matrices:
%\[
%A_1\otimes A_2 = 
%\begin{pmatrix}
%a^1_{1,1}\cdot A^2 & \ldots & a^1_{1,n}\cdot A^2\\
%\ldots & \ldots & \ldots\\
%a^1_{n,1}\cdot A^2 & \ldots & a^1_{n,n}\cdot A_2
%\end{pmatrix}.
%\]
\end{definition}


\subsection{Обобщенные функции}
Понятие обобщенной функции представляет собой расширение понятия функции. Приведем краткое описание основных моментов теории обобщенных функций, взятое из \cite{gelfand}.

\begin{definition}
Основная функция - вещественная функция $f$, имеющая непрерывные производные всех порядков и равная нулю вне некоторой ограниченной области.
\end{definition}
Обозначим через $K$ множество основных функций. Будем говорить, что последовательность $\varphi_1,\varphi_2\ldots$ стремится к нулю в $K$, если все эти функции обращаются в ноль вне одной и той же ограниченной области и равномерно сходятся к нулю вместе со всеми своими производными.

Примером такой последовательности может служить последовательность $\varphi_\nu(x) = \frac{1}{\nu}\varphi(x, a)$, где 
\[
\varphi(x, a) = \left\{
\begin{aligned}
&e^{-\frac{a^2}{a^2 - x^2}},&\ x<a\\
&0,&\ x \geq a\\
\end{aligned} \right. .
\]
Каждая из этих функций и их производных равномерно стремится к нулю, кроме того, каждая из них обращается в ноль при $x \geq a$.

\begin{definition}
Мы говорими, что нам задана обобщенная функция $f$, если определено правило, по которому каждой основной функции $\varphi$ сопоставлено число $(f, \varphi)$, и при этом выполнены условия:
\begin{enumerate}
\item Линейность - $(f, \alpha_1\varphi_1 + \alpha_2\varphi_2) = (f, \alpha_1\varphi_1) + (f, \alpha_2\varphi_2)$;
\item Непрерывность - если некоторая последовательность основных функций $\varphi_1,\varphi_2,\ldots$ стремится к нулю в пространстве $K$, то последовательность чисел $(f, \varphi_1), (f, \varphi_2), \ldots$ стремится к нулю в $\mathbb{R}$.
\end{enumerate}
\end{definition}

Например, рассмотрим функцию $f$, абсолютно интегрируемую в любой конечной области пространства $\mathbb{R}^n$. Она являетс обобщенной, так как с помощью $f$  можно поставить в соответствие число любой основной функции $\varphi$:
\begin{equation}
(f, \varphi) = \int_{\mathbb{R}^n} f(x)\varphi(x) dx. \label{eq:non-singular-generalized}
\end{equation}

Другой пример обобщенной функции - дельта-функция Дирака $\delta(x)$:
\begin{equation}
(\delta, \varphi) = \int_{-\infty}^{+\infty} \delta(x)\varphi(x) dx = \varphi(0). \label{eq:delta-function}
\end{equation}

Обобщенные функции, непредставимые в виде \eqref{eq:non-singular-generalized} называются сингулярными. Дельта функция является примером сингулярных функций.

Каждая сингулярная функция может быть представлена в виде предела последовательности несингулярных обобщенных функций. 
\begin{definition}
Последовательность обобщенных функций $f_1, f_2,\ldots f_\nu,\ldots$ сходится к обобщенной функции $f$, если
\[
\lim_{\nu\to\infty} (f_\nu, \varphi) = (f, \varphi).
\]
\end{definition}

\subsection{Дельта-функция Дирака и ее свойства}

Дельта функция Дирака \eqref{eq:delta-function} может быть представлена как предел последовательности несингулярных функций $f_\nu$, обладающих следующими свойствами:
\begin{enumerate}
\item $\forall M > 0:\ \forall a, b\ |a| \leq M,\ |b| \leq M$ integral 
\[
 \left| \int_a^b f_\nu(\xi)d\xi \right| \leq C(M)
\]
\item Для любых фиксированных $a, b > 0$
\[
\lim_{\nu\to\infty} \int_a^bf_\nu(\xi)d\xi = \left\{
\begin{aligned}
&0,&\ a < b < 0 \mbox{ or } 0 < a < b,\\
&1,&\ a < 0 < b\\
\end{aligned} \right. .
\]
\end{enumerate}
Примером такой функциональной последовательности является последовательность 
\[
f_\epsilon(x) = \dfrac{1}{\pi} \dfrac{\epsilon}{x^2 + \epsilon^2},
\]
где $\epsilon\to 0$.

Дельта функция обладает некоторыми полезными свойствами:
\begin{enumerate}
\item $\int _{-\infty}^{+\infty} f(x)\delta(x - x_0)dx = f(x_0)$,
\item $\delta\left(\dfrac{x}{a}\right) = |a|\delta(x)$,
\item Дельта-функцию $\delta(x)$ можно определить через преобразование Фурье с помощью \eqref{eq:delta-function}:
\[
\tilde{\delta}(\omega) = \int_{-\infty}^{+\infty}\dfrac{e^{i\omega t}}{\sqrt{2\pi}}\delta(t)dt = \dfrac{1}{\sqrt{2\pi}}.
\]
Из полученного выражения следует, что 
\[
\delta(t) = \int_{-\infty}^{+\infty}\dfrac{e^{-i\omega t}}{\sqrt{2\pi}}\tilde{\delta}(\omega)d\omega = \int_{-\infty}^{+\infty}\dfrac{e^{-i\omega t}}{2\pi}d\omega.
\]
\end{enumerate}


\section{Математический формализм квантовой механики}
\subsection{Постулаты квантовой механики}
Аксиоматика квантовой механики может быть сформулирована в виде списка постулатов \cite{Khrennikov_information}, основанном на теории самосопряженных операторов в комплексном гильбертовом пространстве $H$. Нижеизложенный набор постулатов и принципов носит название Копенгагенской интерпретации квантовой механики.
\begin{enumerate}[label=\bfseries Постулат \arabic*:, align=left]
  \item Квантовое состояние $\psi$ представляет собой вектор комплексного гильбертового пространство, такой что $\langle\psi, \psi\rangle = 1$. Состояние квантовой системы полностью описывается этим вектором.
  \item Физическая наблюдаемая $a$ представляется самосопряженным оператором $A$ в гильбертовом пространстве $H$. Различным наблюдаемым соответствуют различные операторы.
  \item Множество значений наблюдаемой, представленной оператором $A$, совпадает со спектром $A$. В случае полностью точечного спектра самосопряженный оператор может быть записанв виде:
  \[
  A = \sum_m a_m\pi_m^A,
  \]
    где $\pi_m^A$ - ортогональный проектор на собственное подпространсво, соответствующее собственному значению $a_m$.
  \item Правило Борна - пусть $A$ - самосопряженный оператор с дискретным спектром, тогда вероятность получить собственное значение $a_m$ в результате измерения определяется правилом:
  \[
  P(a = a_m) = \| \pi_m^a\psi\|^2.
  \]
  \item Постулат фон Неймана - пусть заданы квантовое состояние $\psi$ и самосопряженный оператор $A$. Пространство $H$ представимо в виде $H = H_1\oplus H_2\oplus\ldots H_k$, где $H_i$ - собственное подпространство, соответствующее собственному значению $a_i$. Оператор проектирования $\pi_m^A: H\to H_m$ в этом случае задается соотношением: 
  \[
  \pi_m^A = \sum_{l = 1}^{n_m}\langle\psi,\varphi_{ml}\rangle\varphi_{ml},
  \]
  где $\varphi_{ml}$ - $l$-й собственный вектор из $H_m$.
  Тогда, если в результате измерения $A$  было получено значение $a_m$, то квантовая система оказывается в состоянии $\psi_m$:
  \[
  \psi_m = \frac{\pi_m^A\psi}{ \| \pi_m^a\psi\|}.
  \] 
  \item Эволюция состояния $\psi$ во времени описывается уравнением Шрёдингера:
  \[
  i\hbar \dfrac{d}{dt}\psi(t) = \mathcal{\mathcal{H}}\psi(t)
  \] 
  с начальным состоянием $\psi(0) = \psi_0$, 
  где $\mathcal{H}$  - самосопряженный оператор, описывающий энергию квантовой системы. 
  \item Пусть имеются две квантовые системы, описываемых состояниями из пространств $H_1$ и $H_2$, тогда пространство состояний объединенной системы описывается тензорным произведением $H_1 \otimes H_2$.  
\end{enumerate}
\begin{definition}
Если квантовое состояние объединенной системы $\psi\in H_1\otimes H_2$ не факторизуемо в виде  $\psi = \psi_1\otimes\psi_2$, где $\psi_1\in H_1, \psi_2\in H_2$, тогда оно называется сцепленным или запутанным. 
\end{definition}  
Рассмотрим два гильбертовых пространства $H_1 = H_2$ с размерностью 2 и базисом $\{e_1, e_2\}$. Примером сцепленного состояния в $H = H_1 \otimes H_2$ является следующее состояние: 
\[
\psi = \frac{e_1\otimes e_2 + e_2\otimes e_1}{2}.
\]
Ранее было показано, что в приведенном примере $\psi$ не может быть представлено в виде $\psi = \psi_1\otimes\psi_2$.

Другой пример квантовой запутанности - состояние $\psi = (e_1\otimes e_2 + e_1 \otimes e_1 + 2e_2 \otimes e_1) / \sqrt{6}$.


\subsection{Оператор плотности}

Для описания поведения ансамбля квантовых систем, где каждое состояние реализуется лишь с некоторой вероятностью, мы будем использовать понятия оператора плотности из \cite{Khrennikov_information}.

Для чистого состояния $\psi$ можно определить ортогональный проектор: $P_\psi: P_\psi\varphi = \langle \psi, \varphi \rangle \psi$. Он обладает следующими свойствами:
\begin{enumerate}
\item $P_\psi$ является эрмитовым. 
\begin{proof}
$\langle P_\psi\varphi, v\rangle = \langle \langle\psi,\varphi\rangle \psi, v\rangle = \overline{\langle\psi,\varphi\rangle} \langle\psi, v\rangle = \langle\psi, v\rangle \langle\varphi,\psi\rangle  = \langle\varphi, \langle\psi, v\rangle\psi\rangle = \langle\varphi, P_\psi v \rangle$
\end{proof}
\item $P_\psi \geq 0$.
\begin{proof}
$\langle P_\psi\varphi, \varphi\rangle = \overline{\langle\psi,\varphi\rangle} \langle\psi, \varphi\rangle = |\langle\psi, \varphi\rangle|^2 \geq 0$
\end{proof}
\item $\Tr P_\psi = 1$, где $\Tr A = \sum_k \langle Ae_k, e_k\rangle$ и $\{e_k\}$  - ортогональный базис.
\begin{proof}
$\Tr P_\psi = \sum_{k = 1}^n \langle \langle \psi, e_k\rangle\psi, e_k\rangle = \sum_k \overline{\langle\psi, e_k\rangle}\langle\psi, e_k\rangle = \sum_k\psi_k^2 = 1$
\end{proof}
\item $P_\psi^2 = P_\psi$
\begin{proof}
$P_\psi^2\varphi = \langle\psi, \langle\psi, \varphi\rangle \psi\rangle \psi = \langle\psi, \varphi\rangle \langle\psi, \psi\rangle\psi = \langle\psi, \varphi\rangle\psi = P_\psi\varphi $
\end{proof}
\end{enumerate}
В общем случае для любого состояния или ансамбля состояний можно рассмотреть оператор
\[
\rho = \sum_ip_iP_{\psi_i},
\]
где $p_i$  - вероятность получить состояние $\psi_i$ в результате измерения.

Можно легко показать, что оператор $\rho$ удовлетворяет свойствам 1-3, используя соответствующие свойства 1-3 для каждого из операторов $P_{\psi_i}$. В общем случае свойство 4 для оператора плотности нарушается. Например, рассмотрим оператор $\rho = \cfrac 12P_{\psi_1} + \cfrac 12P_{\psi_2}$, где $\psi_1 = (1, 0)^T$ и $\psi_2 = (0, 1)^T$. Найдем квадрат этого оператора:
\[
\rho^2\varphi = \frac 12\langle\psi_1, \rho\varphi\rangle\psi_1 + \frac 12\langle\psi_2, \rho\varphi\rangle\psi_2 =
\]
\[
= \frac 12\langle\psi_1, \frac 12\langle\psi_1, \varphi\rangle\psi_1 + \frac 12\langle\psi_2, \varphi\rangle\psi_2 \rangle\psi_1 + \frac 12\langle\psi_2, \frac 12\langle\psi_1, \varphi\rangle\psi_1 + \frac 12\langle\psi_2, \varphi\rangle\psi_2 \rangle\psi_2 =
\]
\[
 = 
\frac 14\langle\psi_1, \langle\psi_1,\varphi\rangle\psi_1\rangle\psi_1 + \frac 14\langle\psi_1, \langle\psi_2,\varphi\rangle\psi_2\rangle\psi_1 +
\frac 14\langle\psi_2, \langle\psi_1,\varphi\rangle\psi_1\rangle\psi_2 + \frac 14\langle\psi_2, \langle\psi_2,\varphi\rangle\psi_2\rangle\psi_2
\]
В итоге имеем:
\[
\rho^2\varphi = \frac 14 \langle\psi_1, \varphi\rangle\psi_1 + \frac 14\langle\psi_2, \varphi\rangle\psi_2 = \frac 12\rho \neq \rho.
\]


Оператор плотности $\rho$ также может быть записан в виде $\rho = \sum_i p_i|e_i\rangle\langle e_i|$, гда $|e_i\rangle$ обозначаеи вектор и $\langle e_i|$ обозначает скалярное произведение с вектором $e_i$. 

\subsection{Элементы квантовой теории вероятностей}
Результат любого измерения в квантовой механике зависит от состояния системы до того, как оно изменилось после измерения. Это означает, что значения квантовых вероятностей, математических ожиданий, дисперсий и т.д. также зависят не только от самой наблюдаемой величины, но и от начального состояния, описываемого оператором плотности.

Рассмотрим наблюдаемую $A$ с собственными значениями $\{a_1,\ldots a_n\}$ и собственными векторами $\{f_1,\ldots f_n\}$, а также $\psi$ - состояние системы. Тогда вероятность получить $a_i$ после измерения задается соотношением:
\[
P(A = a_i) = |\langle\psi, f_i\rangle|^2 = \Tr \rho_\psi A_i,
\]
где $A_i = |f_i\rangle \langle f_i|$.
\begin{proof}
$\Tr \rho_\psi A_i = \sum_j \langle\rho_\psi A_if_j, f_j\rangle = \langle \langle\rho_\psi, f_i\rangle, f_i\rangle = \langle \langle \psi, f_i\rangle\psi, f_i\rangle = $\\
$= \overline{\langle\psi, f_i\rangle} \langle\psi, f_i\rangle = |\langle\psi, f_i\rangle|^2.$
\end{proof}
По определению для ансамбля состояний вероятность задается той же формулой:
\[
P(A = a_i) = \Tr \rho |f_i\rangle \langle f_i|.
\]

Квантовое матожидание определяется соотношением:
\[
\overline{A_\rho} \equiv \langle A\rangle = \langle A\rho, \rho\rangle = \Tr\rho A.
\]
Дисперсия в случае квантовой механики определяется так же, как и в классической теории вероятностей:
\[
\sigma^2_{A_\rho} = \overline{\left(A_\rho - \overline{A_\rho}\right)^2} = \overline{A^2_\rho} - \overline{A_\rho}^2.
\] 

\subsection{Принцип неопределенности Гейзенберга}
Согласно принципу неопределенности квантовой механики существует предел точности, с которой одновременно могут быть определени физические наблюдаемые. Этот предел не зависит от использованных при измерении приборов или уровня технологии, он является фундаментальным ограничением и существует при любом измерении.

Впервые принцип неопределенности был сформулирован Вернером Гейзенбергом в 1927 году. Он установил, что чем с большей точность может быть измерена координата частицы, тем с меньшей точностью удастся определить ее импульс.

В 1928 году этот принцип был сформулирован в виде неравенства:
\[
\sigma_x\sigma_p \geq \frac{\hbar}{2},
\]
где $\sigma_x$  - стандартное отклонение для координаты, $\sigma_p$ - стандартное отклонение для импульса и $\hbar$ - постоянная Планка.

В наиболее общем виде принцип неопределенности задается неравенством Шрёдингера:
\begin{equation}
\sigma_A^2\sigma_B^2 \geq \left| \frac{1}{2}\langle\{A, B\}\rangle - \langle A\rangle\langle B\rangle \right|^2 + \left| \frac{1}{2i}\langle [A, B]\rangle\right|^2,
\label{eq:Schrodinger_ineq}
\end{equation}
где $[A, B] = AB - BA$ - оператор, называемый коммутатором, и $\{A, B\} = AB + BA$ - антикоммутатор.

\begin{proof}
Для вывода неравенства Шрёдингера будем использовать неравенство Коши-Шварца для скалярного произведения, известное из функционального анализа:
\begin{equation}
|\langle f, g\rangle |^2 \leq \langle f, f\rangle \langle g, g\rangle. \label{eq:Caushy-Schwartz}
\end{equation}

Квантовая дисперсия $\sigma^2_A$ для самосопряженного оператора $A$ находится по формуле:
\[
\sigma_A^2 = \langle (A - \langle A\rangle)^2\psi, \psi\rangle = \langle  (A - \langle A\rangle)\psi,  (A - \langle A\rangle)\psi\rangle.
\]
Обозначим $f = (A - \langle A\rangle)\psi$ и $g = (B - \langle B\rangle)\psi$, тогда для левой части неравенства \eqref{eq:Caushy-Schwartz} получим:
\[
|\langle f, g\rangle |^2 = |\langle (A - \langle A\rangle)\psi, (B - \langle B\rangle)\psi\rangle |^2 = |\langle (B - \langle B\rangle)(A - \langle A\rangle)\psi, \psi |^2 = 
\]
\[
= | \langle BA\psi, \psi\rangle - \langle B\rangle\langle A\rangle - \langle A\rangle \langle B\rangle + \langle A\rangle\langle B\rangle |^2 = | \langle BA\rangle - \langle A\rangle\langle B\rangle |^2.
\]
Как и любое комплексное число, левая часть неравенства может быть записана в виде:
\[
|\langle f, g\rangle |^2 = \left| \frac{1}{2}(\langle f, g\rangle + \overline{\langle f, g\rangle})\right|^2 + \left| \frac{1}{2i}(\langle f, g\rangle - \overline{\langle f, g\rangle})\right|^2, 
\]
где $\overline{\langle f, g\rangle})$ задается следующим соотношением:
\[
\overline{\langle f, g\rangle}) = \overline{\langle BA\psi, \psi\rangle} - \langle A\rangle\langle B\rangle = \langle\psi, BA\psi\rangle - \langle A\rangle\langle B\rangle = \langle AB\psi, \psi\rangle - \langle A\rangle\langle B\rangle.
\]
Объединая все это вместе в неравенство \eqref{eq:Caushy-Schwartz}, получим искомое неравенство Шрёдингера \eqref{eq:Schrodinger_ineq}:
\[
\left| \frac{1}{2}(\langle BA\rangle - \langle A\rangle\langle B\rangle + \langle AB\rangle - \langle A\rangle\langle B\rangle)\right|^2 + \left| \frac{1}{2i}(\langle BA\rangle - \langle A\rangle\langle B\rangle - \langle AB\rangle + \langle A\rangle\langle B\rangle)\right|^2 \leq \sigma_A^2\sigma_B^2
\]
\[
\Leftrightarrow \left| \frac{1}{2}\langle\{A, B\}\rangle - \langle A\rangle\langle B\rangle \right|^2 + \left| \frac{1}{2i}\langle [A, B]\rangle\right|^2 \leq \sigma_A^2\sigma_B^2
\]
\end{proof}

В квантовой механике оператор координаты частицы задается соотношением:
\[
(\hat{x}f) = x\cdot f(x),
\]
а оператор импульса находится как
\[
(\hat{p}f) = \frac{\hbar}{i}\frac{df}{dx}.
\]
В общем случае из неравенства Шрёдингера  \eqref{eq:Schrodinger_ineq} может быть получено неравенство Робертсона:
\[
\sigma_A^2\sigma_B^2 \geq \frac{1}{4}|[A, B]|^2.
\]

Определим коммутатор операторов координаты и импульса:
\[
[\hat{x}, \hat{p}]f(x) = (\hat{x}\hat{p} - \hat{p}\hat{x})f(x) = x\frac{\hbar}{i}\frac{df(x)}{dx} - \frac{\hbar}{i}\frac{d xf(x)}{dx} = i\hbar If(x),
\]
где $I$ - тождественный оператор. Таким образом, неравенство Робертсона обращается в принцип неопределенности Гейзенберга:
\[
\sigma_A^2\sigma_B^2 \geq \frac{1}{4}\hbar^2.
\]

Из вывода неравенства для принципа неопределенности ясно, что нетривиальное неравенство Робертсона может быть получено для любых операторов $A$ и $B$, имеющих ненулевой коммутатор. В ходе вывода не использовались предположения о технологии или других условиях в ходе измерения, это свидетельствует о том, что полученное неравенством является фундаментальным совйством для некоторых пар операторов, таких как координата и импульс.

\section{Парадокс Эйнштейна-Подольского-Розена}
Парадокс Эйнштейна-Подольского-Розена(ЭПР парадокс, \cite{EPR}) был опубликован в 1935 году в качестве критики некоторых положений Копенгагенской интерпретации квантовой механики, согласно которому любая квантовая система полностью описывается волновой функцией, которая после проведения измерения переходит с некоторой вероятностью в одно из состояний, заданных спектром наблюдаемой. Кроме того, принцип неопределенности Гейзенберга гласит, что координата и импульс частицы не могут быть совместно измерены с хорошей точностью. Эйнштейн и его коллеги были не согласны с вероятностным характером результатов измерений, и их статья стала попыткой показать противоречия, существующие в квантовой механике.

Статья \cite{EPR} основана на мысленном эксперименте, проводимом в следующих условиях. Рассмотрим две системы $S_1$ и $S_2$ с пространствами состояний, равными $L_2(\mathbb{R})$, и источник, производящий частицы для этих систем.  Суперпозиция  $S_1$ и $S_2$ описывается квантовым состоянием из пространства, являющегося тензорным произведением данных гильбертовых пространств.  
Обозначим наблюдаемую в $S_1$ через оператор $A$ с собственными значениями $\{a_k\}$ и собственными векторами $\{\varphi_k(x_1)\}$. Через оператор $B$ обозначим другую наблюдаемую в той же системе, его собственные значения обозначим через $\{b_k\}$, а собственные вектора через $\{\psi_k(x_1)\}$. 

В общем случае для двух гильбертовых пространств $H_1$ и $H_2$ с базисами $\{e_k\}$ и $\{f_k\}$, соответственно, квантовое состояние $\varphi = \varphi_1\otimes\varphi_2$ представимо в виде:
\[
\varphi = \sum_{k, m}c_{k, m}e_k\otimes f_m.
\]
В случае  $H_1 = H_2 = L_2(\mathbb{R})$ тензорное произведение равно $L_2(\mathbb{R})\otimes L_2(\mathbb{R}) = L(\mathbb{R}^2)$. Для данного пространства состояние $\varphi$ представимо в виде:
\[
\varphi (x_1, x_2) = \sum_{k, m}c_{k, m}e_k(x_1)f_m(x_2)
\]
в случае дискретного спектра и в виде:
\[
\varphi (x_1, x_2) = \int u(x, x_1)v(x, x_2)dx.
\]
для непрерывного.

Предположим, что будут проводиться измерения величины $A$. Начальное состояние частицы описывается выражением  $\psi(x_1, x_2) = \sum_k v_k(x_2)\varphi_k(x_1)$. Согластно постулату фон Неймана после измерения состояние системы будет описываться конкретным значением $\psi = \varphi_m(x_1)v_m(x_2)$. Что означает, что после измерений в системе $S_1$, вторая система также окажется в состоянии с определенным значением $v_m(x_2)$. Однако, в условиях, когда две системы  находятся на очень большом расстоянии друг от друга(то есть не могут оказывать влияние друг на друга согласно принципу локальности) это означает, что вторая система находилась в том же состоянии $v_m(x_2)$ и до начала измерений.

Предположим, что вместо того, чтобы искать $A$, было решено проводить измерения величины  $B$ в $S_1$. По аналогии с предыдущим предположением, после измерения система окажется в состоянии $\psi = \zeta_n(x_1)u_n(x_2)$, что означает, что $S_2$ будет иметь состояние с определенным значением $u_n(x_2)$. И в этом случае получается, что вторая система не зависит от проводимых опытов в первой, то есть с самого начала находится в этом состоянии. 

Таким образом, так как измерения в реальности не проводились, можно заключить, что вторая система опиывается волновыми функциями $v_m(x_2)$ и $u_n(x_2)$. Парадокс в том, что можно придумать такие первоначальные состояния, которые не могут совместно быть найденными достаточно точно по принципу неопределенности. Приведем пример такого состояния, приведенной в  указанной статье \cite{EPR}.

Рассмотрим состояние $\psi(x_1, x_2) = \int e^{\frac{ip}{\hbar}(x_1+x_2-x_0)}$, где $x_0$ является константой. С одной стороны, оно может быть записано в виде $\psi(x_1, x_2) = \int\varphi(p, x_1)v(p, x_2)dp$, где $\varphi(p, x_1) = e^{\frac{ip}{\hbar}x_1}$ и $v(p, x_2) = e^{\frac{ip}{\hbar}(x_2-x_0)}$. 

Пусть $A$ - оператор импульса, который задается выражением $\hat{p} = \dfrac{\hbar}{i}\dfrac{d}{dx}$. Для него любому собственному значению $\lambda$ соответствует собственная функция $\psi = e^{\frac{i\lambda}{\hbar}x}$. После измерения $A$ состояние первой системы станет равным его собственной функции $\varphi(p, x_1) = e^{\frac{ip}{\hbar}x_1}$. Состояние второй системы перейдет в $v(p, x_2) = e^{\frac{ip}{\hbar}(x_2-x_0)}$, тоже собственную функцию, соответствующую другому собственному значению, $-p$. А это означает, что вторая система с самого начала была в состоянии с определенным импульсом.

С другой стороны, рассматриваемое начальное состояние представимо в виде: $\psi(x_1, x_2) = \delta(x_1 + x_2 - x_0) = \hbar\int\delta(x - x_1)\delta (x - x_2 + x_0)dx = \int\zeta(x, x_1)u(x, x_2)dx$. 

Пусть $B$ - оператор координаты в первой системе, $B = \hat{x_2}f = x_2f$. Каждому собственному значению оператора $\lambda$ соответствует собственная функция $\delta(x - \lambda)$. После измерения $B$ состояние первой системы перейдет в собственную функцию $\zeta(x, x_1) = \delta(x - x_1)$. Состояние второй системы окажется равным $u(x, x_2) = \delta(x - x_2 + x_0)$, что тоже является собственной функцией оператора координаты, соответствующей значению $x + x_0$. Значит, вторая система с самого начала была в состоянии с определенной координатой.

Исходя из всего вышеперечисленного, получается, что и координата, и импульс второй системы являются элементами реальности, так как измерения в первой системе не могли оказать на нее влияния согласно принципу локальности. Однако, согласно принципу неопределенности, эти величины не могут быть найдены точно одновременно, то есть элементами реальности не являются.

Результатом обозначенного парадокса стал вопрос - действительно ли квантовая механика представляет собой полное описание реальности. Другим возможным решением этого противоречия является отказ от принципа локального реализма.

\section{Неравенства Бэлла}
Одним из решений ЭПР парадокса является то, что квантовая механика не полна, в действительности состояние системы описывается не только волновой функцией  $\psi$, но еще и несколькими скрытыми, то есть пока неизвестными параметрами. В таком случае вероятностный характер предсказаний квантовой механики объясняется существованием нескольких степеней свободы, поэтому реальность может иметь не обязательно вероятностную, но и детерминированную природу.

Джон Бэлл в своей статье \cite{Bell} предположил, что существуют несколько скрытых параметров, обозначенных $\omega$, и результаты измерений представляют собой случайные величины, зависящие от этих параметров. Он сформулировал статистическое неравенство, которое нарушается в предсказаниях квантовой механики. 

В классической теории вероятности ковариация двух случайных величин может быть найдена с использованием формулы:
\[
\langle \xi, \eta \rangle = \int_\Omega \xi(\omega)\eta(\omega)dP(\omega).
\]

\begin{theorem}
Рассмотрим дискретные случайные величины $\xi_a(\omega)$, $\xi_b(\omega)$, $\xi_c(\omega)$, принимающие только значения, равные $\pm 1$. Для них справедливо следующее неравенство:
\[
| \langle\xi_a,\xi_b\rangle -  \langle\xi_c,\xi_b\rangle | \leq 1 - \langle\xi_a,\xi_c\rangle.
\]
\end{theorem}

\begin{proof}
Используя формулу для ковариации, получим:
\[
| \langle\xi_a,\xi_b\rangle -  \langle\xi_c,\xi_b\rangle | = \left| \int_\Omega\xi_a\xi_b dP - \int_\Omega\xi_c\xi_b dP\right| = \left|\int_\Omega (\xi_a - \xi_c)\xi_bdP \right|.
\]
После умножения на $\xi_a^2 = 1$:
\[
| \langle\xi_a,\xi_b\rangle -  \langle\xi_c,\xi_b\rangle | = \left|\int_\Omega (1 - \xi_a\xi_c)\xi_a\xi_bdP \right|.
\]
Используя тот факт, что $|\xi_i| = 1$, получаем правую часть неравенства:
\[
\left|\int_\Omega (1 - \xi_a\xi_c)\xi_a\xi_bdP \right| \leq  \left|\int_\Omega (1 - \xi_a\xi_c)dP\right| = 1 - \langle \xi_a\xi_c\rangle.
\]
\end{proof}

После того, как Бэлл опубликовал свою статью, другими авторами были сформулированы еще несколько неравенств. Рассмотрим некоторые из них. Неравенство Вигнера является более удобным для проведения экспериментов, так как включает в себя соотношение для вероятностей, а не ковариации. 
\begin{theorem}
Для случайных величин, удовлетворяющих условиям предыдущей теоремы, справедливо неравенство:
\[
P(\xi_a = +1, \xi_b = +1) + P(\xi_b = -1, \xi_c = +1) \geq P(\xi_a = -1, \xi_c = +1).
\]
\end{theorem}

\begin{proof}
Первая вероятность может быть записана как
\[
P(\xi_a = +1, \xi_b = +1) = P(\xi_a = +1, \xi_b = +1, \xi_c = +1) + P(\xi_a = +1, \xi_b = +1, \xi_c = -1),
\]
аналогично, запишем вторую как как
\[
P(\xi_b = -1, \xi_c = +1) = P(\xi_a = +1, \xi_b = -1, \xi_c = +1) + P(\xi_a = -1, \xi_b = -1, \xi_c = +1).
\]
Тогда
\[
P(\xi_a = +1, \xi_b = +1) + P(\xi_b = -1, \xi_c = +1) = P(\xi_a = +1, \xi_b = +1, \xi_c = +1) + 
\] 
\[
 + P(\xi_a = +1, \xi_b = +1, \xi_c = -1) + P(\xi_a = +1, \xi_b = -1, \xi_c = +1) + 
\]
\[
 + P(\xi_a = -1, \xi_b = -1, \xi_c = +1) =  P(\xi_a = +1, \xi_c = +1) +
\]
\[
+  P(\xi_a = +1, \xi_b = +1, \xi_c = -1) + P(\xi_a = -1, \xi_b = -1, \xi_c = +1)\geq
\]
\[
 \geq P(\xi_a = +1, \xi_c = +1).
\]
\end{proof}

Еще одним примером неравенства типа Бэлла является неравенство CHSH
(Clauser–Horne–Shimony–Holt).
\begin{theorem}
Для случайных величин $\xi_j(\omega)$ и $\xi'_j(\omega)$, таких что $|\xi_j(\omega)| \leq 1$ и $|\xi'_j(\omega)| \leq 1$ выполняется неравенство:
\[
\langle\xi_1,\xi'_1\rangle + \langle\xi_1,\xi'_2\rangle + \langle\xi_2,\xi'_1\rangle -  \langle\xi_2,\xi'_2\rangle \leq 2
\]
\end{theorem}

\begin{proof}
Для любых действительных чисел, ограниченных единицей, справедливо неравенство:
\[
\xi_1\xi_1' + \xi_1\xi_2' + \xi_2\xi_1' - \xi_2\xi_2' \leq 2.
\]
После интегрирования обеих частей получается неравенство CHSH.
\end{proof}

Основной причиной, почему неравенства типа Бэлла представляют интерес, является тот факт, что они могут быть проверены и проверяются в экспериментах. И так как предсказания квантовой и классической теории не совпадают, можно проверить, чему соответствуют полученные в ходе опыта результаты. Однако, для сравнения двух моделей необходимо иметь механизм связи между ними. Чтобы соотнести выражение из классической теории вероятности и предсказания квантовой механики, Бэллом было сделано несколько предположений, применив которые получается, что неравенства типа Бэлла могут быть нарушены. Проведенные эксперименты также в основном подтверждают нарушение неравенств классической теории вероятностей.

Приведем пример такого нарушения из \cite{Khrennikov_information}. Рассмотрим систему из двух частиц в состоянии $\psi = \frac{1}{\sqrt{2}}(|+-\rangle - |-+\rangle)$ и оператор, соответствующий измерению спина одной из частиц:
\[
\sigma(\theta) = \cos\theta\sigma_z + \sin\theta\sigma_x,
\] 
где $\sigma_x$ и $\sigma_z$ - матрицы Паули:
\[
\sigma_x = 
\begin{pmatrix}
0 & 1\\
1 & 0
\end{pmatrix},\ \sigma_z = 
\begin{pmatrix}
1 & 0\\
0 & -1
\end{pmatrix}.
\]
Для ансамбля частиц оператор $\sigma(\theta) \otimes I$ измеряет спин первой, а оператор $I \otimes \sigma(\theta)$ - второй.
Тогда
\[
P_\psi(\sigma(\theta_1) = + 1, \sigma(\theta_2) = +1) = \cos^2\frac{\theta_1 - \theta_2}{2},
\]
\[
P_\psi(\sigma(\theta_3) = + 1, \sigma(\theta_2) = -1) = \sin^2\frac{\theta_3 - \theta_2}{2},
\]
\[
P_\psi(\sigma(\theta_1) = + 1, \sigma(\theta_3) = +1) = \cos^2\frac{\theta_1 - \theta_3}{2}.
\]
В этом случае неравенство Вигнера принимает следующий вид:
\[
\cos^2\frac{\theta_1 - \theta_2}{2} + \sin^2\frac{\theta_3 - \theta_2}{2} \geq  \cos^2\frac{\theta_1 - \theta_3}{2}.
\]
Возьмем $\theta_1 = 0$, $\theta_2 = 6\theta$, $\theta_3 = 2\theta$ и получим, что следующее неравенство может быть нарушено при достаточно больших $\theta$:
\[
\cos^2 3\theta + \sin^2 2\theta \geq cos^2 \theta.
\]

Нарушение неравенство Бэлла носит название теоремы Бэлла. Если теорема корректна, тогда квантовая механика не полна и/или предположения о локальности или реализме ложны. 

Рассмотрим некоторые популярные интерпретации результатов Бэлла:
\begin{enumerate}
\item Quantum mechanics is complete and nonlocal so it cannot be reduced to the classical theory.
If this interpretation is true then the state of a partial cannot be represented as random variable $\xi_a(\omega)$ so inequalities cannot be applied to the real measurements. This is the most popular interpretation.

\item Quantum mechanics is incomplete and any complete classical theory is nonlocal.
If this interpretation is true then the state of a partial cannot be represented as $\xi_a(\omega)$ because it depends of another partial $b$. The representation as $\xi_{a,b}(\omega)$ doesn't give us the same inequalities so there is no paradox between experiments and classical probability theory.

\item Some of Bell's assumptions about accordance between classical and quantum models are wrong. If this interpretation is true then there is no paradox because its proof is incorrect.
\end{enumerate}

In \cite{Khrennikov_information} it is shown that in Bell's theory there can be some incorrect assumptions in the way of accordance between classical and quantum probabilities.

Firstly, It can be contradicted that classical(an integral) and quantum equalities for covariations are equal.
$$\int_\Omega\xi_a(\omega)\xi_b(\omega)dP_\rho(\omega) \equiv Tr\rho\hat{a}\hat{b}$$
But for other variants of Bell theorem this postulate was replaced by less controversial.

Secondly, domains of classical and quantum variables can be nonequal. There are two systems - the observed and the observer. The probability measure of states for observed partials concerns microscopic world and the observed probabilities concern macroscopic devices. These two systems can have different degrees of freedom, another parameters or possible values. It's hard to determine dependency between them as in theory there is nothing about it.

Moreover, in experimental tests of Bell's inequality statistical data was used. That means that a lot of single experiments were made and their results depended of states of observing devices and assumed hidden variables. So there was different physical context of those experiments. If we fix quantum state $\rho$ it is not necessary that it will always correspond to the fixed classical probability distribution because with hidden variable quantum mechanics is only projection and there is no one to one correspondence. There is one to one correspondence only between classical state $\xi$ and a pair $(\rho, C)$ - quantum state and a context. Using that one can see that Bell's inequality is correct only if contexts of different experiments are the same. Because of many parameters probability to get that is zero. So considering context of experiments Bell's inequality has another form and not violated by experiments.

Another problem with Bell inequalities is experimental data precision. To check something detectors have to have enough efficiency and not give false positive results. Since there is no device without these problems, it is better to have a statistical model which can deal with such experimental errors. One of such models was presented in the Eberhard's article \cite{Eberhard}.

%\section{Eberhard inequality}
%In the paper of Eberhard \cite{Eberhard} a Bell experiment is considered that is performed on entangled states of two photons. For both particles instead of spins polarization measurements are made using Nicol prisms. If photon is polarized horizontally then the ordinary trajectory is applied, otherwise photons polarized in the vertical plane are detected in the extraordinary trajectory.In front of prisms devices are set up that can rotate the plane of photon polarization. The angle of polarization plane rotation of the first particle is denoted by $\alpha$, and for the second particle it is called $\beta$.
%
%Eberhard's approach allows to take into account values of detector efficiency $\eta$ and background noise $\zeta$ before any optimization for such experiment. Here we present a derivation of Eberhard inequality from his paper. 
%
%%It is shown that Bell inequality requires an efficiency $\eta > 2(\sqrt{2} - 1) \approx 82.2\%$. Eberhard inequality still can be used with less efficiency values.
%
%For the described experiment there are four different setups of polarization planes: $(\alpha_1, \beta_1), (\alpha_2, \beta_1), (\alpha_, \beta_2), (\alpha_2, \beta_2)$ where the first item in the pair denotes values of $\alpha$ and the second one $\beta$ values. There are also three possible fates of photon after performing the experiment. Photon that is detected in the ordinary beam is counted with an index (o), photon that is detected in the extraordinary beam is counted with an index (e) and undetected photon is denoted by (u). Therefore for a system with two particles there are nine types of events with can be obtained for every measure.





\bibliography{literature.bib}
\bibliographystyle{ieeetr}
\end{document}